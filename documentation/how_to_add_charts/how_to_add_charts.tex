% author: 'Marie Lohbeck'
% copyright: 'Copyright 2018, Advanced UniByte GmbH'
%
% license notice:
%
% This file is part of PicDat.
% PicDat is free software: you can redistribute it and/or modify it under the terms of the GNU
% General Public (at your option) any later version.
%
% PicDat is distributed in the hope that it will be useful, but WITHOUT ANY WARRANTY; without
% even the implied warranty of MERCHANTABILITY or FITNESS FOR A PARTICULAR PURPOSE. See the GNU
% General Public License for more details.
%
% You should have received a copy of the GNU General Public License along with PicDat. If not,
% see <http://www.gnu.org/licenses/>.

\documentclass[a4paper,11pt]{article}
\usepackage[a4paper, total={6.5in, 9in}]{geometry}
\usepackage[T1]{fontenc}
\usepackage[utf8]{inputenc}
\usepackage{lmodern}
\usepackage{listings}
\usepackage{color}
\usepackage{textcomp}

\renewcommand{\familydefault}{\sfdefault}

\lstset{frame=tb,
  aboveskip=5mm,
  belowskip=5mm,
  showstringspaces=false,
  columns=flexible,
  basicstyle={\tiny\ttfamily},
  breaklines=true,
  breakatwhitespace=true,
  rulecolor=\color{white}
}

\begin{document}
\section*{How to add charts to PicDat}

Adding search keys to PicDat so that they are included in the analysis, and creating new charts can either be very easy or very hard. This depends on how the respective data is constructed, and whether it is similar to data which is already considered. This document tries to explain which options you have for adding additional data.
\bigskip

In any case you need to edit the python source code. You will find two packages, called \verb|perfstat_mode| and \verb|asup_mode|. As PerfStat files and ASUP files are very different (the former are text, the latter are XML), search keys have to be designed for each mode independently. Both modes work with container classes. The docstrings for the classes and the modules containing them give further description. Read them to understand which container is the right one for your use-case.
\bigskip

The general rule is that most container modules are holding constant lists (UPPER\char`_CASE\char`_\break LETTERS) at the top which define the search keys in some way. The format is described locally in the code. If a key about the chart you wish to include can be described in the same way as the keys in one of those lists, you are lucky. Just append it to the respective list. If not, you have to dig deeper into the code.

\subsection*{For PerfStat mode}
In \verb|perfstat_mode|, there are three different classes holding search keys. As PerfStat data is pretty heterogenious, the approach of retreiving data from the keys is fundamentally different among the three classes. You can find the classes inside the modules \verb|per_iteration_container|, \verb|sysstat_container| and \verb|statit_container|.

\subsubsection*{per iteration container}
The \verb|per_iteration_container| has four key lists (and one additional single key). The lists are about the four different objects 'aggregate', 'processor', 'volume', and 'lun'. If you want an additional chart for one of those objects, just append it to the respective list. If you want a chart for similar data, but for a different object, you need to create a new list. Furthermore, you need to append it at least in the following methods: init (declare an emtpy list for the new chart's table), \verb|process_per_iteration_keys| (call \verb|process_object_type| for your object), \verb|rework_per_iteration_data| (flatten your tables if it is not empty), and \verb|get_labels| (append appropriate information to \verb|identifiers|, \verb|units| and \verb|is_histo|). Pay attention to the order in the \verb|flat_tables| list from \verb|rework_per_iteration_data|, it must be equivalent to the order of the list in \verb|get_labels|.

\subsubsection*{sysstat container}
The \verb|sysstat_container| has three key lists. The sysstat blocks in PerfStat files are table-like structures; PicDat reads some of the columns. The considered columns are categorized after the units belonging to them. All considered columns having either the unit '\%' or 'MB/s' or no unit at all. Each of the three key lists in \verb|sysstat_container| is about one of these three categories. The column names are defined in the keys. Values for the same unit can be visualized in the same chart, so each key list is collecting data for exactly one chart. Some of the keys are tuples. This is because the sysstat table headers are spread over two lines and sometimes several keywords are necessary to determine a column. To better understand why the keys have the format they have, just look into a PerfStat file's sysstat block. If you want to evaluate more columns from a sysstat block, try if it makes sense to append to one of the existing key lists which will add a graph line to the respective chart. Otherwise, a more extensive adaption of PicDat is necessary.

\subsubsection*{statit container}
The \verb|statit_container| is responsible for collecting data for a single chart from a specific PerfStat block. This is not really flexible; the container does not even have any kind of key list. So there is no easy way to add a new search key to this container.

\subsection*{For ASUP mode}
Data from ASUP is much more homogenious than PerfStats, so life is somewhat easier with them. As the package \verb|asup_mode| provides support for two different kinds of input data -- XML and JSON -- there are two container classes. Their names are, unsurprisingly, \verb|XmlContainer| inside the \verb|xml_container| module and \verb|JsonContainer| inside the \verb|json_container| module. Both modules have again several lists with search keys. Fortunately, the key lists' structures in both modules are exactly the same. Only the content differs a bit. So, the following applies to both container modules.

\subsection*{search keys}
The key lists \verb|INSTANCES_OVER_TIME_KEYS|, \verb|INSTANCES_OVER_BUCKET_KEYS|, and \verb|COUNTERS_|\break\verb|OVER_TIME_KEYS| are each for a different kind of chart. The difference is the charts' x-axis, and whether each individual graph line in one chart is for different instances or different counters. Again, read the list descriptions in the code. Additionally, the following drawing might help your imagination:

\begin{lstlisting}
          INSTANCES_OVER_TIME_KEYS                         INSTANCES_OVER_BUCKET_KEYS                        COUNTERS_OVER_TIME_KEYS

    ^                                                ^                                                ^                                                         
    |                                                |                                                |                                                         
    |                                instance1       |                                instance1       |                                counter1                
    |                                instance2       |                                instance2       |                                counter2                
    |                                instance3       |                                instance3       |                                counter3                
  v |                                   ...        v |                                   ...        v |                                   ...                   
  a |                                              a |                                              a |                                                         
  l |                                              l |                                              l |                                                         
  u |                                              u |                                              u |                                                         
  e |                                              e |                                              e |                                                         
    |                                                |                                                |                                                         
    |                                                |                                                |                                                         
    |                                                |                                                |                                                         
    |                                                |                                                |                                                         
    ------------------------------->                 ------------------------------->                 ------------------------------->                          
        time                                             bucket                                           time                                                  
\end{lstlisting}
If one of the lists fits your needs, append your key and your chart will be drawn. Otherwise, you will need to reimplement most of the methods. 
\bigskip

When you add keys to the \verb|json_container|, be aware that you probably \textbf{also need to change the config.yml for the convert\_ccma\_to\_json script!} Otherwise, PicDat will search for your new objects and counters in your JSON files, but has no chance to find anything, because you didn't ask Trafero to convert this information at all.
\bigskip

Finally, it should be mentioned that there is a third container module named \verb|hdf5_container| inside the asup mode package. It is deprecated, but it still works. You can manipulate its keys in the same way as them in the other two asup container modules.

\subsection*{calculate further charts}
The general approach of PicDat is to collect and visualize data given in some kind of performance files. But sometimes, it is also helpful to do some math with the collected values before visualising them. For example, PicDat performs a unit conversion with some charts.

Beyond that, there are other use cases to perform some more complex math. The use-case that this section deals with is when you want a chart whose values are not from the original performance data, but can be caluculated from it. As example there is the chart \textit{aggregate: free\_space\_fragmentation} which represents the quotient \textit{user\_writes/cp\_reads}. 'free\_space\_fragmentation' itself is not a counter from the PerfStat performance files, but 'user\_writes' and 'cp\_reads' are.
\bigskip

If you want to include such a new, calculated chart, you have to add some code to the method \verb|calculate_further_charts()| in \verb|xml_container| or \verb|json_container|. This code needs to create a new \verb|Table| object, which holds the data for your new chart. It's up to you to fill the table with the values you want. \verb|calculate_further_charts()| gets called \textit{after} PicDat searched the performance files, so you can refer to all other tables. The function \verb|table.do_table_operation| might be useful.

In order for PicDat to handle your newly created table, you have to add it to the container's dict \verb|self.tables|. Furthermore, add the corresponding unit to \verb|self.units|, if applicable. Also, you need to append the dict key you used for \verb|self.tables| and \verb|self.units| to the list \verb|FURTHER_CHARTS| at the top of the container module. This is important, because outer modules obtain information on which 'further' charts to create from this list, analog to how they do for each key list. This is also why each element of the \verb|FURTHER_CHARTS| list must observe a specific format: It has to be a tuple of two strings.
\bigskip

There are plenty of code comments for the \verb|FURTHER_CHARTS| list and the \verb|calculate_further_|\break\verb|charts()| method. In order to calculate a new chart, follow the example given in \verb|calculate_further_|\break\verb|charts()|.

\end{document}
